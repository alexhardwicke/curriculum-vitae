%-----------------------------------------
% Alex Hardwicke
% CV
%
% URL: http://github.com/mdwrigh2/resume
% This document is covered under the Creative Commons Attribution 3.0 Unported License
%-----------------------------------------

%!TEX TS-program = lualatex
%!TEX encoding = UTF-8 Unicode

\documentclass[11pt,a4paper,final]{moderncv}
\usepackage{fontspec}
\moderncvtheme[blue]{classic}
\newfontface\lserif{Liberation Serif}

% DOCUMENT LAYOUT
\usepackage[scale=0.88]{geometry}
\setlength{\hintscolumnwidth}{20mm}
\AtBeginDocument{\recomputelengths}
% FONTS
\usepackage[latin1]{inputenc}
\defaultfontfeatures{Mapping=tex-text} % converts LaTeX specials (``quotes'' --- dashes etc.) to unicode

% Remove % to set to different fonts
%\setromanfont [Ligatures={Common},Numbers={OldStyle}]{Adobe Caslon Pro}
%\setmonofont[Scale=0.8]{Monaco}

% ---- CUSTOM AMPERSAND
%\newcommand{\amper}{{\fontspec[Scale=.95]{Adobe Caslon Pro}\selectfont\itshape\&}}
\newcommand{\Csh}{C{\lserif\#}}
\newcommand{\Fsh}{F{\lserif\#}}

% ---- MARGIN YEARS
%\newcommand{\years}[1]{\marginpar{\scriptsize #1}}


% PDF SETUP
% ---- FILL IN HERE THE DOC TITLE AND AUTHOR


% Personal Information
\firstname{Alexander}
\familyname{Hardwicke}
\address{Magistratsvägen 17D}{226 43 Lund}
\mobile{0762 480546}
\email{alex.hardwicke@outlook.com}
\homepage{www.alexhardwicke.com}

\title{Alex Hardwicke's CV}

\nopagenumbers{}

\begin{document}

\maketitle

\section{Kunskaper}
    \cvline{expert:}{\Csh{}, WinRT/Phone/UWP}
    \cvline{duktig:}{Android, ASP.NET, \Fsh, Git, Java, JavaScript, MSSQL}
    \cvline{intresserad:}{Swift, TypeScript}

\section{Arbetserfarenhet}
    \cventry{okt. 2014\\-- nu}{Systemutvecklare}{Erisma}{Malmö, Sverige}{}{
        \begin{itemize}% \itemsep -2pt % reduce space between items
            \item Utvecklade Erismas core produkter.
            \item Ansvarig för alla core produkter efter 6 månaders anställning.
            \item Förbättrade utvecklarmiljön i företaget, genom att bland annat lägga till stöd for felspårning i alla Erismas produkter, migrera från .NET 3.5 till .NET 4.5, flytta företaget från mercurial till git och implementera ett modernt buildsystem.
            \item Utvecklade och körde interna utbildingar i git och \Csh{}.
        \end{itemize}
    }
    \cventry{aug. 2014\\-- okt. 2014}{Lärare}{Edument}{Malmö, Sverige}{}{
        \begin{itemize}% \itemsep -2pt % reduce space between items
            \item Utvecklade två intensivkurser i java på grund- och medelnivå.
            \item Utförde utbildning inom git.
        \end{itemize}
    }
    \cventry{sep. 2012\\-- maj 2014}{Labbassistent}{Malmö Högskola}{Malmö, Sverige}{}{
        \begin{itemize}% \itemsep -2pt % reduce space between items
            \item Assistent till förstaårsstudenter som läste datavetenskap.
            \item Assistent till förstaårsstudenter som läste datateknik.
            \item Assistent till andra- och tredjeårsstudenter (mina klasskamrater) som läste Androidutveckling.
        \end{itemize}
    }
    
\section{Utbildning}
    \cventry{sep. 2011\\-- juni 2014}{Datavetenskap}{}{Malmö Högskola}{Malmö, Sverige}{
    Högska betyg i alla kurser.}

\section{Egna projekt}
    \cventry{dec. 2013\\-- nu}{Surge}{Windows 8.1 \& 10}{}{\Csh{} \& \Fsh{}}{
        \begin{itemize}% \itemsep -2pt % reduce space between items
            \item Fjärrkontroll/gränssnitt för Transmission applikationen.
            \item Över 25000 nerladdningar.
            \item Open source. Koden finns på \url{https://github.com/alexhardwicke/Surge}
            \item Mer detaljer finns på \url{http://www.alexhardwicke.com/surge}
            \item Ladda ner från \url{http://apps.microsoft.com/windows/app/surge/97e772a6-62ff-4921-9cbf-44f9ec186037} \newline
        \end{itemize}
    }
    \cventry{juni 2012\\-- augusti 2012}{Promenade}{Android}{}{Java}{
        \begin{itemize}% \itemsep -2pt % reduce space between items
            \item Min första stora mobilapplikation som jag skrev för att lära mig mobilutveckling med Android.
            \item Mer detaljer finns på \url{http://www.alexhardwicke.com/promenade}
            \item Open source. Koden finns på \url{https://github.com/alexhardwicke/promenade}
            \item Ladda ner från \url{https://play.google.com/store/apps/details?id=com.digitalpies.promenade}
        \end{itemize}
    }

\section{Annat}
    \cvlistitem{Medlem av StackOverflow: \url{http://stackoverflow.com/users/2079472/alex-hardwicke}}
    \cvlistitem{Jag pratar både engelska (mitt modersmål) och svenska flytande.}
\end{document}